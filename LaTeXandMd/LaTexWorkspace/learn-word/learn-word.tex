% -*- coding: utf-8 -*-
% !TEX program = xelatex

\documentclass[12pt,notheorems,xcolor={rgb}]{beamer}

\PassOptionsToPackage{BoldFont}{xeCJK} % ctex 2.4.4 取消了伪粗体
%\usepackage{xeCJK}
%\xeCJKsetemboldenfactor{5}
\usepackage[UTF8,noindent]{ctex}
\setCJKsansfont{SimHei}
\setCJKmonofont{SimHei}
\renewcommand\CJKfamilydefault{\CJKsfdefault}

\usetheme{dirichlet}

\setbeamersize{text margin left=4mm,text margin right=4mm}

\AtBeginSubsection{}

\usepackage{arev}

\renewcommand{\baselinestretch}{1.1} % ctex 2.4.1 开始为 1,之前为 1.3

%% LaTeX 中 默认 \parskip=0pt plus 1pt,而 Beamer 中默认 \parskip=0pt

%% \parskip 用 plus 1fil 没有效果,用 plus 1fill 则节标题错位
\setlength{\parskip}{5pt plus 1pt minus 1pt}       % 段间距为 5pt + 1pt - 1pt
%\setlength{\baselineskip}{19pt plus 1pt minus 1pt} % 行间距为 5pt + 1pt - 1pt
\setlength{\lineskiplimit}{4pt}                    % 行间距小于 4pt 时重新设置
\setlength{\lineskip}{4pt}                         % 行间距太小时自动改为 4pt

\AtBeginDocument{
  \setlength{\baselineskip}{19pt plus 1pt minus 1pt} % 似乎不能放在导言区中
  \setlength{\abovedisplayskip}{4pt minus 2pt}
  \setlength{\belowdisplayskip}{4pt minus 2pt}
  \setlength{\abovedisplayshortskip}{2pt}
  \setlength{\belowdisplayshortskip}{2pt}
}

\newfontfamily{\adventor}{TeX Gyre Adventor}
\setbeamerfont{title}{size=\Huge,series=\bfseries,family=\adventor\youyuan}

\setbeamerfont{section title}{size=\LARGE}

\usepackage{multicol}

\usepackage{listings}

\makeatletter
\AtBeginDocument{%
  \def\lst@visiblespace{%
    {\color{text1!70!back1}\lst@ttfamily{\char32}\textvisiblespace}%
  }%
}
\makeatother

\lstset{
  basicstyle=\ttfamily\color{accent2},
  literate=*{\&}{{\color{accent1}\&}}{1}{@}{{\color{accent1}@}}{1},
  showspaces=true,
  escapechar=\%,
  %escapechar=`,
  %frame=single,
  %backgroundcolor=\color{lightgray},
  %xleftmargin=8pt,
  %framexleftmargin=8pt
}

\let\lst=\lstinline

\newcommand{\rightkey}{\textcolor{text1!70!back1}{$\rightarrow$}}

%\lstMakeShortInline` % FIXME

\usepackage{tcolorbox}

\tcbset{
  standard jigsaw,
  opacityback = 0,
  colframe = text1!70!back1,
  coltext  = accent3,
  boxrule  = 0.2mm,
  left     = 2mm,
  right    = 2mm,
  top      = 2mm,
  bottom   = 2mm,
  middle   = 2mm
}

% 左右并排例子框
\newtcolorbox{exampleh}[1][0.5]{
  sidebyside = true,
  lefthand ratio = #1,
}

% 上下排列例子框
\newtcolorbox{examplev}{
  sidebyside = false
}

\usepackage{tabularx}
\usepackage{tabu}

\newenvironment{multieqns}{
  \setlength{\arraycolsep}{0.14em}
  \begin{array}{*{12}{rc}}
}{
  \end{array}
}

% http://tex.stackexchange.com/q/66949/8956
\DeclareTextCommandDefault{\nobreakspace}{\leavevmode\nobreak\ }

\newenvironment{framex}{\begin{frame}[fragile=singleslide,environment=framex]}{\end{frame}}

% 默认是 \raggedright,改为两边对齐
\usepackage{ragged2e}
\justifying
\let\oldraggedright\raggedright
\let\raggedright\justifying

\def\lead#1{\textcolor{accent1}{#1}}
\def\bold#1{\textcolor{accent2}{#1}}
\def\warn#1{\textcolor{accent3}{#1}}

\def\code#1{\textcolor{accent2}{#1}}
\def\demo#1{\textcolor{accent3}{#1}}

\newcommand\zz[1]{\text{\bold{\string#1}}}

\newcommand{\mcmd}[1]{\bold{\string#1}&\warn{$#1$}}

\newcommand{\out}[1]{{$\color{accent3}#1$}}

\begin{document}

\title{用 Word 排版数学文档}
\author{\href{https://lvjr.bitbucket.io}{吕荐瑞}}
\institute{暨南大学数学系}

\begin{frame}[plain]
\titlepage
\end{frame}

\section{开始使用}

\subsection{新版公式}

\begin{frame}
\frametitle{新版公式}
从 Microsoft Word 2007 开始,微软提供了新的数学公式功能.相对于 MathType 插件写出的公式,它有如下优点:
\begin{itemize}
  \item 技术先进,排版出的数学公式更美观
  \item 兼容良好,在不同的电脑上表现一致
  \item 使用方便,无需安装插件就可以使用
  \item 输入快捷,可以纯粹用键盘输入公式
\end{itemize}
\end{frame}

\subsection{插入公式}

\begin{framex}
\frametitle{插入公式}
在新版 Word 中,要插入一个数学公式,有如下两种方法:
\begin{enumerate}
  \item 使用键盘快捷键“Alt+=”
  \item 在插入菜单中点击“公式”项
\end{enumerate}
\end{framex}

\subsection{两种公式}

\begin{framex}
\frametitle{两种公式}
在新版 Word 中,公式有\bold{内嵌公式}和\bold{显示公式}两种,两种公式有一些区别:
\begin{itemize}
  \item 显示公式单独一行居中显示,而内嵌公式与文本放在同一行
  \item 显示公式中的分式一般比内嵌公式的大
\end{itemize}
而一个公式属于哪种类型,由该行是否有其它内容自动决定:
\begin{itemize}
  \item 你在某些文字后面直接按“Alt+=”将得到内嵌公式
  \item 而新起一行后再按 “Alt+=”则得到自动居中的显示公式
\end{itemize}
点击公式右下角的三角形按钮,可以改变公式的类型.
\end{framex}

\section{单行公式}

\subsection{上标下标}

\begin{framex}
\frametitle{上标下标}
在 Word 的数学公式中,上标和下标分别用 \lst!^! 和  \lst!_! 表示.比如
\begin{exampleh}
\lst!x_n ! \tcblower $x_n$
\end{exampleh}
\begin{exampleh}
\lst!x^2 ! \tcblower $x^2$
\end{exampleh}
在上面两个例子中,左边表示输入的方式,右边表示得到的结果.
\par
注意输入最后的 \lst! ! 表示空格符.
空格符标示了某部分输入的结束,从而 Word 将会自动构建此部分公式.
\end{framex}

\begin{framex}
\frametitle{上标下标}
在输入时,并不需要在公式的每一部分后面都加上空格符.比如
\begin{examplev}
\lst!(x_1+x_2)^2=x_1^2+2x_1 x_2+x_2^2 !
\tcblower
$(x_1+x_2)^2=x_1^2+2x_1x_2+x_2^2$
\end{examplev}
在这个例子可以看到,括号和运算符也会自动标示某部分公式的结束,因此不需要加上 \lst! ! 空格符.
\end{framex}

\begin{framex}
\frametitle{上标下标}
如果上下标中包含多个符号,一般要将它们放在一对圆括号中.
\begin{exampleh}
\lst!a^(p-1) ! \tcblower $a^{p-1}$
\end{exampleh}
\begin{exampleh}
\lst!a^p-1! \tcblower $a^p-1$
\end{exampleh}
如上所述,运算符会导致 Word 构建之前的上标,因此结果不同.
\end{framex}

\subsection{数学分式}

\begin{framex}
\frametitle{数学分式}
在 Word 中输入分式很自然,比如
\begin{exampleh}
\lst!(a+b)/c ! \tcblower $\frac{a+b}{c}$
\end{exampleh}
若需要斜线形式的分式,可以输入 \lst!\/!.比如
\begin{exampleh}
\lst!(a+b)\/c ! \tcblower $(a+b)/c$
\end{exampleh}
\end{framex}

\subsection{排版根号}

\begin{framex}
\frametitle{排版根号}
要得到二次根号,可以用 \lst!\sqrt! 命令.下面两种输入的结果相同
\begin{exampleh}
\lst!\sqrt 5 ! \tcblower $\sqrt{5}$
\end{exampleh}
\begin{exampleh}
\lst!\sqrt(5) ! \tcblower $\sqrt{5}$
\end{exampleh}
但用第二种方法还可以得到一般的根号.比如
\begin{exampleh}
\lst!\sqrt(n&x) ! \tcblower $\sqrt[n]{x}$
\end{exampleh}
三次根号和四次根号也可以分别用 \lst!\cbrt! 和 \lst!\qdrt! 命令.
\end{framex}

\subsection{分隔括号}

\begin{framex}
\frametitle{分隔括号}
圆括号 \lst!()!,方括号 \lst![]! 和花括号 \lst!{}! 称为\bold{分隔符}.
它们的大小将根据其中内容的大小自动调整.比如
\begin{exampleh}
\lst!(1+1/n)^n ! \tcblower $\left(1+\frac1n\right)^n$
\end{exampleh}
注意不要忘记在后面输入空格符.左右括号可以不必一样,比如
\begin{exampleh}
\lst!(1/3 ,1/2] ! \tcblower $\left(\frac13,\frac12\right]$
\end{exampleh}
\end{framex}

\subsection{特殊字符}

\begin{framex}
\frametitle{特殊字符}
在数学公式中,很少会用到 \_ 和 \^{} 字符.
确实需要时,可以分别输入  \lst!\_! 和 \lst!\^! 得到.
其他特殊字符输入方式类似.比如
\begin{exampleh}
\lst!\^ \_ \( \)! \tcblower \^{} \_ ( )
\end{exampleh}
此时得到的括号将不被当作分隔符,从而不会自动调整大小。
\end{framex}

\section{多行公式}

\subsection{数学阵列}

\begin{framex}
\frametitle{数学阵列}
利用 \lst!\matrix! 命令可以得到数学阵列.比如
\begin{exampleh}
\lst!\matrix(1&2@5&10) !
\tcblower
$\begin{array}{cc}
1 & 2 \\
5 & 10
\end{array}$
\end{exampleh}
其中 \lst!&! 表示下一列,而 \lst!@! 表示下一行,每列都居中对齐.\par
再加上括号就可以得到矩阵.比如(注意后面的两个空格符)
\begin{exampleh}
\lst!(\matrix(1&2@5&10) ) !
\tcblower
$\left(\begin{array}{cc}
1 & 2 \\
5 & 10
\end{array}\right)$
\end{exampleh}
\end{framex}

\begin{framex}
\frametitle{数学矩阵}
在 Word 2013 中用 \lst!\pmatrix! 命令也可以得到矩阵.比如
\begin{exampleh}[0.7]
\lst!\pmatrix(1&2&3@4&5&6@7&8&9) !
\tcblower
$\left(\begin{array}{ccc}
1 & 2 & 3 \\
4 & 5 & 6 \\
7 & 8 & 9
\end{array}\right)$
\end{exampleh}
其中 \lst!&! 表示下一列,而 \lst!@! 表示下一行,每列都居中对齐.
\end{framex}

\subsection{多行公式}

\begin{framex}
\frametitle{多行公式}
利用 \lst!eqarray! 命令,很容易写出多行的对齐公式.比如
\begin{examplev}
\lst!\eqarray((x+y)^2&(x+y)(x+y)@&x^2+2xy+y^2) !
\tcblower
\begin{align*}
(x+y)^2 &= (x+y)(x+y) \\
        &= x^2 + 2xy + y^2
\end{align*}
\end{examplev}
其中  \lst!&! 指明各行对齐的位置.各行之间用 \lst!@! 隔开.
\end{framex}

\begin{framex}
\frametitle{多行公式}
利用 \lst!eqarray! 写出的多行公式,可以包含多个对齐位置.比如
\begin{examplev}
\lst!\eqarray(a&=1,\ensp&b&=-2,@c&=-3,\ensp&d&=4) !
\tcblower
\begin{alignat*}{4}
a&=1,  & \quad b&=-2, \\
c&=-3, & \quad d&=4.
\end{alignat*}
\end{examplev}
如果使用若干个  \lst!&! 将公式隔开为多块,各行的奇数块均\warn{右}对齐,偶数块均\warn{左}对齐.
其中 \lst!\ensp! 命令用于插入额外空白,见第 \pageref{space} 页。%FIXME
\end{framex}

\subsection{括号匹配}

\begin{framex}
\frametitle{括号匹配}
利用 \lst!eqarray! 命令也可以得到对齐的方程组.比如
\begin{examplev}
\lst!{\eqarray(3x&+&y&=&7@x&+&5y&=&11) \close  !
\tcblower
$$\left\{\begin{multieqns}
3x &+& y &=& 7 \\ x &+& 5y &=& 11
%10&x+&3&y=2 \\ 3&x+&13&y=4
\end{multieqns}\right.$$
\end{examplev}
由于左右配对的分隔符才会自动调整大小,在只有单边括号时候,
可以用 \lst!\open! 或者 \lst!\close! 命令分别添加左右空白分隔符.
\end{framex}

\subsection{分段函数}

\begin{framex}
\frametitle{分段函数}
利用 \lst!eqarray! 命令也可以得到分段函数.比如
\begin{examplev}
\lst!{\eqarray(&x&," "x>=0;@-&x&," "x<0.) \close  !
\tcblower
$$\left\{\begin{aligned}
 &x, &  x \ge 0; \\
-&x, &  x < 0.
\end{aligned}\right.$$
\end{examplev}
这里用英文双引号将空格符包含以来,以免它们被  Word 忽略.
\end{framex}

\begin{framex}
\frametitle{分段函数}
在 Word 2013 中用 \lst!\cases! 命令也可以得到分段函数.比如
\begin{examplev}
\lst!|x|=\cases(&x&," if "x>=0;@-&x&," if "x<0.) !
\tcblower
$$|x|=\left\{\begin{array}{l@{}l}
\phantom{-}x, & \text{\ if\ } x \ge 0; \\
          -x, & \text{\ if\ } x < 0.
\end{array}\right.$$
\end{examplev}
公式中的文字也用英文双引号包含起来,以免被认为是数学符号.
\end{framex}

\section{函数算符}

\subsection{数学函数}

\begin{framex}
\frametitle{数学函数}
Word 可以识别常见的数学函数,比如 \lst!exp!,\lst!ln!,\lst!sin!,\lst!arctan!,
\lst!max!,\lst!det!,\lst!deg!,\lst!gcd!,\lst!dim!,\lst!sup! 等,并用直立字体显示它们.
\begin{exampleh}[0.58]
\lst!sin 2x!\rightkey\lst!=k sin x!\rightkey\lst!cos x!\rightkey
\tcblower
$\sin 2x = k\sin x\cos x$
\end{exampleh}
\begin{enumerate}
  \item 要让 Word 识别出数学函数,函数名必须和前面的字母隔开(前面为数字或运算符时不用隔开)
  \item 在函数名后面按空格符 \lst! ! 将得到一个虚线方框表示该函数的参数,之后输入的字符都会放入其中.
  \item 在输入完函数的参数后,要按右方向键 \rightkey{} 离开参数方框.
\end{enumerate}
\end{framex}

\begin{framex}
\frametitle{数学函数}
如果把函数参数放在括号里面,此时无需按方向键,输入更简单.
\begin{exampleh}[0.55]
\lst!sin^2 (x) cos^2 (x) =1!
\tcblower
$\sin^2(x) + \cos^2(x) =1$
\end{exampleh}
但注意右括号后要加上空格,以让配对括号自动调整大小.
\end{framex}

\begin{framex}
\frametitle{数学函数}
也有方法可以避免按方向键,而且函数参数最外面无需加上括号.
\begin{exampleh}[0.6]
\lst!sin\begin \pi/2 \end  =1!
\tcblower
$\sin\frac{\pi}{2}=1$
\end{exampleh}
这个例子使用 \lst!\begin! 和 \lst!\end! 命令将函数参数正确地分隔出来.
在它们后面按空格键后 \lst!\begin! 和 \lst!\end! 变成空心的方括号符号.
之后再按空格键以结束函数参数.因此 \lst!\end! 后面要两个空格符.
\end{framex}

\subsection{极限运算}

\begin{framex}
\frametitle{极限运算}
极限运算是一种带参数的运算,输入方法和数学函数类似.比如
\begin{exampleh}[0.65]
\lst!lim_(x->0) sin x!\rightkey\lst!/x !\rightkey\lst!=1!
\tcblower
$\lim\limits_{x\to0}\dfrac{\sin x}{x}=1$
\end{exampleh}
在输入上面第一个空格后同样出现一个虚线方框,用于包含极限运算的函数.
因此在写完 $\frac{\sin x}{x}$ 后也要按右方向键 \rightkey{} 结束该函数.
\end{framex}

\begin{framex}
\frametitle{极限运算}
在显示公式中,极限的下标出现在底部;而在内嵌公式中,极限的下标出现在右下角.
比如 \demo{$\lim_{x\to0}\frac{\sin x}{x}=1$}.这种形式不太好看.
\par
要保证 $x\to0$ 总是出现在底部,可以用底标命令 \lst!\below!.比如
\begin{examplev}
\lst!"lim"\below(x->0)  \begin sin x!\rightkey\lst!/x \end  =1!
\tcblower
$\lim\limits_{x\to0}\tfrac{\sin x}{x}=1$
\end{examplev}
上面用双引号包含 \lst!lim!,以让 Word 把它当作普通文本的底标.
如果极限函数不是分式,\lst!\begin! 和 \lst!\end! 也可以省略.
\end{framex}

\subsection{求和连乘}

\begin{framex}
\frametitle{求和连乘}
求和运算和连乘运算分别用 \lst!\sum! 和 \lst!\prod! 命令来表示.
它们的输入方式和极限运算类似,只是它们可以有上标.比如
\begin{examplev}
\lst!\sum_(n=1)^(\infty) 1/n^2  !\rightkey\lst!=\pi^2 /6 !
\tcblower
$$\sum\limits_{n=1}^{\infty} \dfrac{1}{n^2}  =\dfrac{\pi^2}{6}$$
\end{examplev}
\begin{exampleh}[0.69]
\lst!\prod_(p<=x) (1-1/p )^(-1) !\rightkey
\tcblower
$$\prod\limits_{p\le x}\left(1-\frac1p\right)^{-1}$$
\end{exampleh}
\end{framex}

\begin{framex}
\frametitle{求和连乘}
在内嵌公式中,求和与连乘的上下标总出现在右边.比如 \demo{$\sum_{n=1}^k\frac{1}{n}$}.
\par
要保证上下标始终出现在顶部和底部,可以用顶标命令 \lst!\above! 和底标命令 \lst!\below!.比如
\begin{examplev}
\lst!"\sum"\below(n=0) \above(k)  \begin 1/n \end  !
\tcblower
$\sum\limits_{n=1}^k\tfrac{1}{n}$
\end{examplev}
这里用双引号包含 \lst!\sum!,以让 Word 把它当作普通文本的顶标和底标.
如果求和式不是分式,\lst!\begin! 和 \lst!\end! 也可以省略.
\end{framex}

\section{命令速查}

\subsection{自动更正}

\begin{framex}
\frametitle{自动更正}
某些数学符号可以通过输入两三个字符,自动合并得到.比如\par
\catcode`\~=12
\begin{tabularx}{\textwidth}{|*{3}{XX|}}
  \hline
  \lst!...! & \out{\ldots} &
  \lst!+-!  & \out{\pm} &
  \lst!-+!  & \out{\mp} \\
  \lst!~=!  & \out{\cong} &
  \lst!<<!  & \out{\ll} &
  \lst!>>!  & \out{\gg} \\
  \lst!->!  & \out{\rightarrow} &
  \lst!<=!  & \out{\leq} &
  \lst!>=!  & \out{\geq} \\
  \hline
\end{tabularx}
\end{framex}

\subsection{空白调整}

\begin{framex}
\frametitle{空白调整}\label{space}
数学公式中输入的多余空格符很多时候将会被忽略.
若要在公式中添加额外的空白,可以用下面这些命令:\par
\renewcommand{\arraystretch}{1.2}
\begin{tabularx}{\textwidth}{XXX}
  \hline
  \zz{\ensp}     & $9/18$ em & \demo{$|\!\mkern9mu|$} \\
  \hline
  \zz{\vthicksp} & $6/18$ em & \demo{$|\!\,\,|$} \\
  \hline
  \zz{\thicksp}  & $5/18$ em & \demo{$|\!\;|$} \\
  \hline
  \zz{\medsp}    & $4/18$ em & \demo{$|\!\:|$} \\
  \hline
  \zz{\thinsp}   & $3/18$ em & \demo{$|\!\,|$} \\
  \hline
  \zz{\hairsp}   & $1/18$ em & \demo{$|\!\mkern1mu|$} \\
  \hline
\end{tabularx}\par
其中  1em 等于字母 M 的宽度,各命令所得宽度如最右栏所示.
\end{framex}

\subsection{希腊字母}

\begin{frame}
\frametitle{希腊字母}
%小写希腊字母:\par
\unskip
$\begin{tabu}{|*{3}{X>{\color{accent3}}l|}X[1.3]>{\color{accent3}}l|}
\hline
\zz{\alpha}  & \alpha  & \zz{\mu}      & \mu      & \zz{\epsilon} & \epsilon & \zz{\varepsilon} & \varepsilon \\
\zz{\beta}   & \beta   & \zz{\nu}      & \nu      & \zz{\theta}   & \theta   & \zz{\vartheta}   & \vartheta \\
\zz{\gamma}  & \gamma  & \zz{\xi}      & \xi      & \zz{\kappa}   & \kappa   &                  & \\
\zz{\delta}  & \delta  & \zz{\tau}     & \tau     & \zz{\pi}      & \pi      & \zz{\varpi}      & \varpi \\
\zz{\zeta}   & \zeta   & \zz{\upsilon} & \upsilon & \zz{\rho}     & \rho     & \zz{\varrho}     & \varrho \\
\zz{\eta}    & \eta    & \zz{\chi}     & \chi     & \zz{\sigma}   & \sigma   & \zz{\varsigma}   & \varsigma \\
\zz{\iota}   & \iota   & \zz{\psi}     & \psi     & \zz{\phi}     & \phi     & \zz{\varphi}     & \varphi \\
\zz{\lambda} & \lambda & \zz{\omega}   & \omega   &               &          &                  & \\
\hline
\end{tabu}$
\vfill
%\par 大写希腊字母 :\par
$\begin{tabu}{|*{4}{X>{\color{accent3}}l|}}
\hline
\zz{\Gamma} & \Gamma & \zz{\Lambda} & \Lambda & \zz{\Sigma}   & \Sigma   & \zz{\Psi}   & \Psi \\
\zz{\Delta} & \Delta & \zz{\Xi}     & \Xi     & \zz{\Upsilon} & \Upsilon & \zz{\Omega} & \Omega \\
\zz{\Theta} & \Theta & \zz{\Pi}     & \Pi     & \zz{\Phi}     & \Phi     &             & \\
\hline
\end{tabu}$
\end{frame}

\subsection{数学符号}

\begin{frame}
\frametitle{数学符号}
\renewcommand{\arraystretch}{1.2}%
\begin{tabu}{|*{3}{Xl|}}
  \hline
  \mcmd{\ge}     &  \mcmd{\cdot}    & \mcmd{\pm}         \\
  \mcmd{\le}     &  \mcmd{\cdots}   & \mcmd{\times}      \\
  \mcmd{\neq}    &  \mcmd{\mid}     & \mcmd{\div}        \\
  \mcmd{\sim}    &  \mcmd{\partial} & \mcmd{\leftarrow}  \\
  \mcmd{\approx} &  \mcmd{\infty}   & \mcmd{\rightarrow} \\
  \mcmd{\cong}   &  \mcmd{\int}     & \mcmd{\Leftarrow}  \\
  \mcmd{\equiv}  &  \mcmd{\iint}    & \mcmd{\Rightarrow} \\
  \hline
\end{tabu}
\end{frame}

\subsection{参考资料}

\begin{frame}
\frametitle{参考资料}
\begin{description}[一二三四]
  \item[网页说明] \href{http://www.unicode.org/notes/tn28/}{http://www.unicode.org/notes/tn28/}.
  \item[详细文档] \href{http://www.unicode.org/notes/tn28/UTN28-PlainTextMath-v3.pdf}
                      {http://www.unicode.org/notes/tn28/\newline UTN28-PlainTextMath-v3.pdf}.
\end{description}
\end{frame}

\end{document}
