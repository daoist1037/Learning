\documentclass[letterpaper]{article}

% \usepackage[T1]{fontenc}
% fontenc 宏包是用来配合传统的LATEX 字体的,
% fontspec 宏包调用ttf 或otf 格式字体,就不要再使用fontenc 宏包。
\begin{document}
	{\small The small and\textbf{bold} Romans ruled}
	\par 
	{\Large all of great big{\itshape Italy}.}
	
	
	% 字号
	He likes {\LARGE large and{\small small} letters}.
	
	% 如果不是在导言区全局修改,而想要局部地改变某个段落的行距,需要用\selectfont 命令使\linespread 命令的改动立即生效:
	{\linespread{2.0}\selectfont The baseline skip is set to be twice the normal baseline skip.Pay attention to the \verb|\par| command at the end. \par}
	In comparison, after the curly brace has been closed,everything is back to normal.\par 

	% 以下长度分别为段落的左缩进、右缩进和首行缩进:它们和设置行距的命令一样,在分段时生效。
	% \setlength{\leftskip}{20pt}  不知道怎么用
	% \textsc{\setlength{\rightskip}{20pt}}   不知道怎么用
	% \setlength{\parindent}{1em}
	\noindent sdjhfjskgffffff

	% 水平间距
	% \hspace 命令生成的水平间距如果位于一行的开头或末尾,则有可能因为断行而被“吞掉”。可使用\hspace* 命令代替\hspace 命令得到不会因断行而消失的水平间距。
	This\hspace{1.5cm}is a space of 1.5 cm.

	% 命令\stretch{⟨n⟩} 生成一个特殊弹性长度,参数⟨n⟩ 为权重。它的基础长度为0pt,但可以无限延伸,直到占满可用的空间。如果同一行内出现多个\stretch{⟨n⟩},这一行的所有可用空间将按每个\stretch 命令给定的权重⟨n⟩ 进行分配。
	% 命令\fill 相当于\stretch{1}5
	x\hspace{\stretch{1}}
	x\hspace{\stretch{3}}
	x\hspace{\fill}x

	% \vspace 的间距在一页的顶端或底端可能被“吞掉”,类似\hspace 在一行的开头和末尾那样。对应地,\vspace* 命令产生不会因断页而消失的垂直间距。\vspace 也可用\stretch 设置无限延伸的垂直长度。
	% \vspace 也可以在段落内使用
	Use command \verb|\vspace| to add \vspace{12pt} some spaces between lines in a paragraph.
	% \bigskip, \medskip, \smallskip 来增加预定义长度的垂直间距
	\parbox[t]{3em}{TeX\par TeX}	%parbox垂直盒子
	\parbox[t]{3em}{TeX\par\smallskip TeX}
	\parbox[t]{3em}{TeX\par\medskip TeX}
	\parbox[t]{3em}{TeX\par\bigskip TeX}

	\twocolumn
	sadddddd\newpage sddddddddddd
	\thispagestyle{empty}
\end{document}