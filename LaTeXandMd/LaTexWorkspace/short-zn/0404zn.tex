\documentclass{article}
\usepackage{ulem}
\title{标题}
\author{Mary\thanks{E-mail:*****@***.com}\and Ted\thanks{Corresponding author}\and Louis}


\begin{document}
    \maketitle
    % 摘要
    
    \begin{abstract}
		这是一篇关于勾股定理的小短文
		grap
		实时编译结果查看
		撒娇的哈师大
	\end{abstract}
	\LaTeX
    \textbackslash      %输出斜杠
    \par
    It’s difficult to find \ldots .
    \par 
    It’s dif{}f{}icult to f{}ind \ldots .   %避免连字?貌似无效

    \par 
    % 连字号- 用来组成复合词;短破折号– 将数字连接表示范围;长破折号— 作为破折号使用
    daughter-in-law, X-rated\\
    pages 13--67\\
    yes---or no?

    \par 
    % 省略号 下面等效
    \ldots  \dots

    \par 
    % 波浪号
    a\~{}z \qquad a$\sim$z

    \par 
    % 特殊西文符号与重音
    % 这里用 \‘  \’会出错,不知道原因
    H\^otel, na\"\i ve, \'el\'{e}ve,\\
    sm\o rrebr\o d, !Se\ norita!,\\
    Sch\"onbrunner Schlo\ss{}
    Stra\ss e

    \par 
    % 下划线
    An \underline{underlined} text.
    An example of \underline{somelong and underlined words.}
    % 查看两者区别 前者不同的单词可能生成高低各异的下划线,并且无法换行;后者能够轻松生成自动换行的下划线
    An example of \uline{somelong and underlined words.}

    \par 
    % ulem包重定义了emph为下划线,不导包时为  强调,即正常字体中用斜体强调,斜体中正常字体来强调
    Some \emph{emphasized words,including \emph{double-emphasized}words}, are shown here.

    \par 
    haaa aaaa aaaaasddd ddddddddddddd dddaaaaaaaa aaaaaaaaaaa  dsaasda s 
    Fig.~2a \\  %  ~  生成不会断行的空格
    \newline    %常用于文本段落,而双斜杠多用于数学公式
    Donale~E. Knuth

    % \newpage \clearpage 都可用于分页;前者在双栏中对一栏起作用,且两者在浮动体的操作上也不同
    \newpage
    as 
    \linebreak[3]
    % 用数字⟨n⟩ 代表适合/不适合的程度,取值范围为0-4,不带可选参数时,缺省为4
    % 1 必须不换行  4 必须换行  2 3 可能换行,强调级别不同
    % \nolinebreak[n]反之
    % \pagebreak[n] 分页  
    % \nopagebreak[n]
    bs
    \clearpage
    as

    \par
    % 如果一些单词没能自动断词,我们可以在单词内手动使用\- 命令指定断词的位置:
    % 如果没遇到换行的情况,就不起作用;
    I think this is: su\-per\-cal\-i\-frag\-i\-lis\-tic\-ex\-pi\-al\-i\-do\-cious.
    % \documentclass[11pt,a4paper]{article}   %指定文档类型
\setlength{\textwidth}{14.5cm} % 设置正文的宽度
\setlength{\textheight}{20.5cm} % 设置正文的高度
\usepackage{amssymb} % 调用宏包,提供更多数学符号
\usepackage{CJK}	%提供中文支持 已经不被推荐
\usepackage{amsmath}	%提供更多数学公式环境
\begin{document}    %正文开始
    Hi,this is my first \LaTeX\ life.\\
    \linebreak[4]   %建议分页[n] n=0 1 2 3 4 建议力度依次增大
    This is \textbf bold face style.\\  %\textbf 只对后面第一个字符起作用
    This is \textbf{bold face} style.\\ %用分组扩展\textbf 的作用
    This is \bfseries bold face style.\\    %\bfseries 对后面所有字符有效
    This is {\bfseries bold face} style.\\    %用分组限制\bfseries的作用
    \newpage	%强制分页
    \pagebreak[2]   %同\linebreak[n]类似,但用于建议分页
    % \enlargethispage*{10mm}
    The Euler equation is given by\\[3mm]	%增加当前行与下一行的距离
    $$e^{ix} \triangleq \cos(x) + i\sin(x)$$
    \\*     %换行,但是禁止在分页时换行
\end{document}  %正文结束
\end{document}