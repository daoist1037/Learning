\documentclass{article}
\usepackage{amsmath}
\usepackage{amssymb}
\usepackage{amsthm}

% 其中带星号的命令定义带上下限的算符
\DeclareMathOperator{\argh}{argh}
\DeclareMathOperator*{\nut}{Nut}
\begin{document}
	Add $a$ squared and $b$ squared
	to get $c$ squared. Or, using
	a more mathematical approach:
	$a^2 + b^2 = c^2$

	Add $a$ squared and $b$ squared
	to get $c$ squared
	\begin{equation}
	a^2 + b^2 = c^2
	\end{equation}
	Einstein says
	\begin{equation}
	E = mc^2 \label{clever}
	\end{equation}
	This is a reference to
	\eqref{clever}.


	It’s wrong to say
	\begin{equation}
	1 + 1 = 3 \tag{dumb}
	\end{equation}
	or
	\begin{equation}
	1 + 1 = 4 \notag
	\end{equation}


	Again\ldots
	\begin{equation*}
	a^2 + b^2 = c^2
	\end{equation*}
	or you can type less for the
	same effect:
	\[ a^2 + b^2 = c^2 \]
	or if you like the long one:
	\begin{displaymath}
	a^2 + b^2 = c^2
	\end{displaymath}


	In text:
	$\lim_{n \to \infty}\sum_{k=1}^n \frac{1}{k^2}= \frac{\pi^2}{6}$.\\
	In display:
	\[
	\lim_{n \to \infty}
	\sum_{k=1}^n \frac{1}{k^2}
	= \frac{\pi^2}{6}
	\]


	$f''(x)$

	% 分式使用\frac{分子}{分母} 来书写。分式的大小在行间公式中是正常大小,而在行内被极度压缩。amsmath 提供了方便的命令\dfrac 和\tfrac,令用户能够在行内使用正常大小的行间公式,或是反过来
	In display style:
	\[
	3/8 \qquad \frac{3}{8}
	\qquad \tfrac{3}{8}
	\]
	In text style:
	$1\frac{1}{2}$~hours \qquad
	$1\dfrac{1}{2}$~hours


	% 特殊的分式形式,如二项式结构,由amsmath 宏包的\binom 命令生成:
	Pascal’s rule is
	\[
	\binom{n}{k} =\binom{n-1}{k}
	+ \binom{n-1}{k-1}
	\]

	% 关系符
	$\ne  \le \ge \approx$
	$\neq \leq \geq$

	\[
	f_n(x) \stackrel{*}{\approx} 1
	\]

	$\times \div \cdot \pm \mp \nabla \partial$

	$\ker \dim \hom \deg$
	$\limsup \liminf \sup \det \Pr \gcd$

	$a\bmod b\\x\equiv a\pmod{b}$


	\[\argh 3 = \nut_{x=1} 4x\]%自定义的算符

	In text:
	$\sum_{i=1}^n \quad
	\int_0^{\frac{\pi}{2}} \quad
	\oint_0^{\frac{\pi}{2}} \quad
	\prod_\epsilon $ \\
	In display:
	\[\sum_{i=1}^n \quad
	\int_0^{\frac{\pi}{2}} \quad
	\oint_0^{\frac{\pi}{2}} \quad
	\prod_\epsilon \]
	% 可以在巨算符后使用\limits 手动令上下限显示在上下方,\nolimits 则相反
	In text:
	$\sum\limits_{i=1}^n \quad
	\int\limits_0^{\frac{\pi}{2}} \quad
	\prod\limits_\epsilon $ \\
	In display:
	\[\sum\nolimits_{i=1}^n \quad
	\int\limits_0^{\frac{\pi}{2}} \quad
	\prod\nolimits_\epsilon \]

	% amsmath 宏包还提供了\substack,能够在下限位置书写多行表达式;subarray 环境更进一步,令多行表达式可选择居中(c) 或左对齐(l):
	\[
	\sum_{\substack{0\le i\le n \\
	j\in \mathbb{R}}}
	P(i,j) = Q(n)
	\]
	\[
	\sum_{\begin{subarray}{l}
	0\le i\le n \\
	j\in \mathbb{R}
	\end{subarray}}
	P(i,j) = Q(n)
	\]

	% 数学重音和上下括号
	$\bar{x}_0 \quad \vec{x}_1 \quad \hat{x}_2$\\
	$\underline{\underline{1/3}} = 0.\overline{3}$\\
	$\hat{XY} \quad \widehat{XY} \quad \overrightarrow{AB}$\\
	% \overbrace 和\underbrace 命令用来生成上/下括号,各自可带一个上/下标公式
	$\underbrace{\overbrace{a+b+c}^6 \cdot \overbrace{d+e+f}^7}_\text{meaning of life} =42$

	% 为箭头增加上下标
	\[ a\xleftarrow{x+y+z} b \]
	\[ c\xrightarrow[x<y]{a*b*c}d \]


	% 使用\left 和\right 命令可令括号(定界符)的大小可变,
	% \left 和\right 必须成对使用。需要使用单个定界符时,另一个定界符写成\left. 或\right.
	\[1 + \left(\frac{1}{1-x^{2}}
	\right)^3 \qquad
	\left.\frac{\partial f}{\partial t}
	\right|_{t=0}\]

	% 用\big、\bigg等命令生成固定大小的定界符 (\bigl 和\bigr 不必成对出现)
	% \left 和\right 分界符包裹的公式块是不允许断行的 而\big 和\bigg 等命令不受限制
	$\Bigl((x+1)(x-1)\Bigr)^{2}$\\
	$\bigl( \Bigl( \biggl( \Biggl( \quad
	\bigr\} \Bigr\} \biggr\} \Biggr\} \quad
	\big\| \Big\| \bigg\| \Bigg\| \quad
	\big\Downarrow \Big\Downarrow
	\bigg\Downarrow \Bigg\Downarrow$


	% 多行公式的首行左对齐,末行右对齐,其余行居中
	\begin{multline}
	a + b + c + d + e + f
	+ g + h + i \\
	= j + k + l + m + n\\
	= o + p + q + r + s\\
	= t + u + v + x + z
	\end{multline}


	% 给等号后添加一对括号{} 以产生正常的间距
	% 多个公式多个编号
	\begin{align}
	a ={} & b + c \\
	={} & d + e + f + g + h + i
	+ j + k + l \notag \\
	& + m + n + o \\
	={} & p + q + r + s
	\end{align}
	% 下面没有对齐
	\begin{gather}
	a = b + c \\
	d = e + f + g \\
	h + i = j + k \notag \\
	l + m = n
	\end{gather}
	% 以上带星号的命令均表示没有编号


	% 多个公式一个编号 与equation 环境套用。
	% aligned 等环境也可以用定界符包裹。
	\begin{equation}
	\begin{aligned}
	a &= b + c \\
	d &= e + f + g \\
	h + i &= j + k \\
	l + m &= n
	\end{aligned}
	\end{equation}


	% 数组可作为一个公式块,在外套用\left、\right 等定界符
	\[ \mathbf{X} = \left(
	\begin{array}{cccc}
	x_{11} & x_{12} & \ldots & x_{1n}\\
	x_{21} & x_{22} & \ldots & x_{2n}\\
	\vdots & \vdots & \ddots & \vdots\\
	x_{n1} & x_{n2} & \ldots & x_{nn}\\
	\end{array} \right) \]

	\begin{equation}
	\begin{aligned}
	|x| = \left\{
		\begin{array}{rl}
		-x & \text{if } x < 0,\\
		0 & \text{if } x = 0,\\
		x & \text{if } x > 0.
		\end{array} \right.
	\end{aligned}
	\end{equation}

		\begin{equation}
		\begin{aligned}
|x| =
\begin{cases}
-x & \text{if } x < 0,\\
0 & \text{if } x = 0,\\
x & \text{if } x > 0.
\end{cases}
		\end{aligned}
		\end{equation}

	% 不带定界符的matrix,以及带各种定界符的矩阵pmatrix(()、bmatrix([)、Bmatrix({)、vmatrix(|)、VmVmatrix(||)用这些环境时,无需给定列格式
	\[
	\begin{matrix}
	1 & 2 \\ 3 & 4
	\end{matrix} \qquad
	\begin{bmatrix}
	x_{11} & x_{12} & \ldots & x_{1n}\\
	x_{21} & x_{22} & \ldots & x_{2n}\\
	\vdots & \vdots & \ddots & \vdots\\
	x_{n1} & x_{n2} & \ldots & x_{nn}\\
	\end{bmatrix}
	\]
	% 在矩阵中的元素里排版分式时,一来要用到\dfrac 等命令,二来行与行之间有可能紧贴着,这时要用到3.6.6 小节的方法来调节间距:
	\[
	\mathbf{H}=
	\begin{bmatrix}
	\dfrac{\partial^2 f}{\partial x^2} &
	\dfrac{\partial^2 f}
	{\partial x \partial y} \\[8pt]
	\dfrac{\partial^2 f}
	{\partial x \partial y} &
	\dfrac{\partial^2 f}{\partial y^2}
	\end{bmatrix}
	\]

	\newpage
	%间距
	\begin{equation}
		\begin{aligned}
			&aa
			&a\ a\\
			&a\quad a\\
			&a\qquad a\\
			&a\,a\\
			&a\:a\\
			&a\;a\\
			&a\!a\\
		\end{aligned}
	\end{equation}
	$$\int \iint \iiint \iiiint$$



	% 字体
	$\mathcal{R} \quad \mathfrak{R}
	\quad \mathbb{R}$
	\[\mathcal{L}
	= -\frac{1}{4}F_{\mu\nu}F^{\mu\nu}\]
	$\mathfrak{su}(2)$ and
	$\mathfrak{so}(3)$ Lie algebra


	% \displaystyle 行间公式尺寸	\textstyle 行内公式尺寸		\scriptstyle 上下标尺寸	\scriptscriptstyle 次级上下标尺
	\[
	P = \frac
	{\sum_{i=1}^n (x_i- x)(y_i- y)}
	{\displaystyle \left[
	\sum_{i=1}^n (x_i-x)^2
	\sum_{i=1}^n (y_i-y)^2
	\right]^{1/2} }
	\]

	% 公式粗体
	$\mu, M \qquad
	\boldsymbol{\mu}, \boldsymbol{M}$

	% 定理环境
	\newtheorem{mythm}{My Theorem}[section]
	\begin{mythm}\label{thm:light}
	The light speed in vaccum
	is $299,792,458\,\mathrm{m/s}$.
	\end{mythm}
	\begin{mythm}[Energy]
	The relationship of energy,
	momentum and mass is
	\[E^2 = m_0^2 c^4 + p^2 c^2\]
	where $c$ is the light speed
	described in theorem \ref{thm:light}.
	\end{mythm}

	% amsthm包
	\theoremstyle{definition} \newtheorem{law}{Law}
	\theoremstyle{plain} \newtheorem{jury}[law]{Jury}
	\theoremstyle{remark} \newtheorem*{mar}{Margaret}
	\begin{law}\label{law:box}
	Don’t hide in the witness box.
	\end{law}
	\begin{jury}[The Twelve]
	It could be you! So beware and
	see law~\ref{law:box}.\end{jury}
	\begin{jury}
	You will disregard the last
	statement.\end{jury}
	\begin{mar}No, No, No\end{mar}
	\begin{mar}Denis!\end{mar}
	% 证毕
	\begin{proof}
	For simplicity, we use
	\[
	E=mc^2
	\]
	That’s it.
	\end{proof}
	% 如果行末是一个不带编号的公式, 符号会另起一行,这时可使用\qedhere 命令将符号放在公式末尾:
	\begin{proof}
	For simplicity, we use
	\[
	E=mc^2 \qedhere
	\]
	\end{proof}
	% 在使用带编号的公式时,建议最好不要使用\qedhere 命令,而是让proof 环境自动生成

\end{document}
