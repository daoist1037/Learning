%导言区
%一般只有10 11 12pt
\documentclass[12pt]{ctexbook}%article,book,report,letter,document
\setlength{\textwidth}{11.5cm} % 设置正文的宽度
\setlength{\textheight}{20.5cm} % 设置正文的高度
\usepackage{ctex}

\usepackage{graphicx}%pictures
\graphicspath{.}
\usepackage{amsmath}
\newcommand{\degree}{^\circ}
\title{\heiti 我的第一个LaTeX文档}
\author{\kaishu 李国庆}
\date{\today}
%正文区
\begin{document}
	\maketitle %letter 不能有这个命令
	\tableofcontents
	\chapter{绪论}
	\section{研究目的}
	\section{研究现状}
	%篇章结构 
	\section{引言}%小节
	这是正文合法的
	\chapter{实验}
	\section{实验方法}
	\section{实验结果}
	\subsection{数据}%子小节
	\subsection{图表}
	\subsubsection{实验条件}
	\subsection{结果分析}
	\includegraphics[scale=0.17]{a}%缩放因子
	
	\begin{tabular}{|l ||c|c| c| p{1cm}|} %l left  c centre  r right 
		\hline
		name & chinese & math & english & text\\
		\hline \hline
		张三 & 87 & 100 & 93 & excellent\\
		\hline
	\end{tabular}\\
	插图\ref{fig} %交叉引用就可以显示插图的编号
	\begin{figure}[htbp]
		\centering
		\includegraphics[scale=0.17]{a}%缩放因子
		\caption{2233}\label{fig}
	\end{figure}
	\begin{table}
		\begin{tabular}{|l ||c|c| c| p{1cm}|} %l left  c centre  r right 
			\hline
			name & chinese & math & english & text\\
			\hline \hline
			张三 & 87 & 100 & 93 & excellent\\
			\hline
		\end{tabular}
	\end{table}
	a\quad b\\
	a\qquad b\\
	a\,b a\thinspace b\\%约1/6个em
	a\enspace b\\%0.5个空格
	a~b\\%硬空格 中文符号
	a\kern -1em b\\这三个都是指定宽度
	a\hskip 1em b\\
	a\hspace{35pt} b\\
	a\hphantom{xyz}b %宽度为三个占位字符的宽度
    a\hfill b%弹性长度的宽度
    \S \P \dag \ddag \copyright \pounds 
    \TeX{} \LaTeX{} \LaTeXe{}\\
    ` ' `` ''
    	
	Hello world!\\ \\
	%字体样式设置
	\textrm{Roman Family}\\
	Roman family
	{\rmfamily Roman family}\\
	Roman family\\
	Let $f(x)$ be defined by formula\\ \\
	%字体 粗细宽度
	\textmd{Medium Series}		\textmd{Boldface Series}\\
	{\mdseries Medium Series}	{\bfseries Boldface Series}\\ \\
	%字体形状 直立 斜体 伪斜体 小型大写
	\textup{Upright Shape}  \textit{Italic Shape}\\
	\textsl{Slanted Shape}  \textsc{Small Shape}\\
	{\upshape Upright Shape} {\itshape Italic Shape}\\
	{\slshape Slanted Shape} {\scshape Small Shape}\\ \\
	%中文字体设置
	{\songti 宋体} \quad {\heiti 黑体} \quad {\fangsong 仿宋} \quad {\kaishu 楷书}\\
	中文字体的\textbf{粗体 是用黑体表示} \textit{斜体 是用楷书表示}\\ \\
	%字体大小
	{\tiny		 	hello}\\
	{\scriptsize 	hello}\\
	{\footnotesize  hello}\\
	{\small			hello}\\
	{\normalsize	hello}\\
	{\large			hello}\\
	{\LARGE			hello}\\
	{\huge			hello}\\
	{\Huge			hello}\\ \\
	%中文字号设置命令
	{\zihao{2} 你好!}
	\begin{eqnarray}
	f(x)&=x_1^2+x_2^3+5\\
	C_{ij}&=\{0,1,-1\}_{n\times n}
	\end{eqnarray}
	\begin{eqnarray}\setcounter{equation}{7}
	f(x)&=x_1^2+x_2^3+5\\
	C_{ij}&=\{0,1,-1\}_{n\times n}
	\end{eqnarray}
	\begin{equation}\setcounter{equation}{7}
		f(x)=x_1^2+x_2^3+5\\
	\end{equation}
	\begin{equation}
	C_{ij}=\{0,1,-1\}_{n\times n}
	\end{equation}\\
	$\angle C=90\degree$
\end{document}